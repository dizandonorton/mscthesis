\begin{abstract}
\setstretch{1.5}
\ac{GP} is a \ac{ML} technique used in optimization problems where one tries to find the best solution on
a set of possible solutions. \ac{GP} is part of the \ac{EC} paradigm inspired by the theory of natural
evolution of species to guide the search of solutions.

In this work, we evaluated the performance of \ac{GP} on the problem of prediction of pharmacokinetic parameters used in the drug development 
process. This is an optimization problem where, given a set of drug molecular descriptors and the corresponding values of the pharmacokinetic 
parameters or the molecular activity, \ac{GP} is then used to build a mathematical model that estimates such values. To this end, we used data of drugs 
with known values of some pharmacokinetic parameters. To evaluate \ac{GP} performance in solving the problem at hand, several \ac{GP} models were 
implemented with different fitness functions and configurations.

The results from the different GP models were compared with the results currently published in the literature, and they confirm that \ac{GP} is a 
promising technique from the point of view of the accuracy of the solutions, their generalization ability and the correlation between the 
predicted and the actual values.
\end{abstract}