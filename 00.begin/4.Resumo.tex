\begin{resumo}
\setstretch{1.5}
A \ac{PG} é uma técnica de Aprendizagem de Máquina (\ac{ML}) aplicada em problemas de otimização onde pretende-se achar
a melhor solução num conjunto de possíveis soluções. A \ac{PG} faz parte do paradigma conhecido por \ac{CE} que tem como inspiração 
à teoria da evolução natural das espécies para orientar a pesquisa das soluções.

Neste trabalho, é avaliada a performance da \ac{PG} no problema de previsão de parâmetros farmacocinéticos utilizados no processo 
de desenvolvimento de fármacos. Este é um problema de otimização onde, dado um conjunto de descritores moleculares de fármacos e os valores
correspondentes dos parâmetros farmacocinéticos ou de sua atividade molecular, utiliza-se a \ac{PG} para construir uma função matemática 
que estima tais valores. Para tal, foram utilizados dados de fármacos com os valores conhecidos de alguns parâmetros 
farmacocinéticos. Para avaliar o desempenho da \ac{PG} na resolução do problema em questão, foram implementados diferentes modelos de
\ac{PG} com diferentes funções de \emph{fitness} e configurações.

Os resultados obtidos pelos diferentes modelos foram comparados com os resultados atualmente publicados na literatura e os mesmos confirmam que a 
\ac{PG} é uma técnica promissora do ponto de vista da precisão das soluções encontradas, da capacidade de generalização e da correlação 
entre os valores previstos e os valores reais.
\end{resumo}