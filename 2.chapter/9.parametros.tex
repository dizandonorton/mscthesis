\section{PARÂMETROS}
\label{sec:2parametros}

Antes de executar a \ac{PG}, é necessário executar os seguintes passos preparatórios:

\begin{itemize}
  \item {Definir o conjunto de terminais $T$}
  \item {Definir o conjunto de funções $F$}
  \item {Escolher a função de \emph{fitness}}
  \item {Definir os parâmetros para controlar execução}
  \item {Escolher o critério de paragem}
\end{itemize}

Por outro lado, os parâmetros para controlar a execução da \ac{PG} são os seguintes:

\begin{itemize}
  \item {Tamanho da população}
  \item {Técnica utilizada para a inicialização da população}
  \item {Algoritmo de seleção}
  \item {Método e taxa de cruzamento}
  \item {Método e taxa de mutação}
  \item {Profundidade máxima das árvores}
\end{itemize}

O critério de paragem é o método que determina o resultado (fim) da execução. A execução pode ser terminada quando for atingido o número máximo de 
gerações ou quando é encontrado um indivíduo com uma \emph{fitness} considerada aceitável. A escolha destes parâmetros é um passo muito importante 
uma vez que os mesmos determinam a performance da \ac{PG}. A definição dos parâmetros escolhidos para o presente trabalho são apresentados no 
capítulo \ref{cap:5}.