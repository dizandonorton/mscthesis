\section{MUTAÇÃO}
\label{sec:2mutacao}

Outro operador que altera a estrutura de um indivíduo é a mutação. Em \ac{PG} o método mais comum de mutação é a mutação de subárvore que
seleciona aleatóriamente um ponto (nó) numa árvore e substitui a subárvore com raíz nesse ponto por uma outra subárvore gerada aleatóriamente.
A \figref{Figura271} ilustra um exemplo de um mutação de subárvore.

\begin{figure}[H]
	\centering
	\begin{forest}
		[, phantom, s sep = 2.5cm,
			[{Pai}, baseline, for children={no edge}
				[$+$,circle,draw,
					[$*$,circle,draw
						[$x_1$,circle,draw]
						[$2$,circle,draw]
					]
					[,phantom]
					[$-$,circle,draw
						[$x_1$,circle,draw]
						[$x_2$,circle,draw,fill=gray]
					]
				]
			]
			[{Subárvore gerada aleatoriamente}, baseline, for children={no edge}, below = 5.5cm
				[$+$,name=subarvore,circle,draw,tikz={\node [draw,red,fit to tree,dashed] {};}
					[$/$,circle,draw
						[$x_1$,circle,draw]
						[$x_2$,circle,draw]
					]
					[,phantom]
					[$4$,circle,draw
						[$x_1$,circle,draw]
						[$x_2$,circle,draw]
					]
				]
			]
			[{Filho}, baseline, for children={no edge}, below = 2.25cm
				[$+$,circle,draw,
					[$*$,circle,draw
						[$x_1$,circle,draw]
						[$2$,circle,draw]
					]
					[,phantom]
					[$*$,circle,draw
						[$x_1$,circle,draw]
						[$+$,circle,name=fpm,draw
							[$/$,circle,draw
								[$x_1$,circle,draw]
								[$x_2$,circle,draw]
							]
							[,phantom]
							[$4$,circle,draw
								[$x_1$,circle,draw]
								[$x_2$,circle,draw]
							]
						]
					]
				]
			]
		]
		\draw[arrows={-triangle 45},dashed,color=red] (subarvore) to [out=south east,in=south west] (fpm);
	\end{forest}
	\caption{Exemplo de mutação de subárvore. O nó cinzento é o ponto de mutação e é substituído pela subárvore gerada aleatoriamente}
	\label{Figura271}
\end{figure}

Outros operadores de mutação também muito utilizados em \ac{PG} são: mutação de troca e mutação de ponto. A mutação por troca seleciona
aleatoriamente duas subárvores e as troca. A mutação de ponto seleciona aleatóriamente um nó e o susbtitui com um nó aleatório de mesma aridade.