\section{INTRODUÇÃO}
\label{sec:2introducao}

A \ac{PG} é uma técnica utilizada em \ac{CE} para a resolução de problemas de pesquisa e otimização. 
Em \ac{PG}, diferentemente de uma pesquisa aleatória, o objetivo é fazer com que as soluções melhorem ao longo da execução do
algoritmo através da aplicação de uma função de \emph{fitness}. Durante este processo, os indivíduos mais aptos (ou parte das suas 
características), são preservados através da utilização de operadores genéticos \citep{Koza1992}. No geral, a \ac{PG} cria 
programas de computador para resolver problemas executando os seguintes passos:

\begin{enumerate}
	\item{Cria uma população de programas de computador (soluções, indivíduos).}
  	\item{Executa iterativamente os seguintes passos até que um critério de paragem seja satisfeito:}
  	\begin{enumerate}
    	\item{Executa cada programa na população e atribui um valor de \emph{fitness} de acordo a sua capacidade de resolver
    	o problema.}
    	\item{Cria uma nova população aplicando as seguintes operações:}
    	\begin{enumerate}
    		\item{Seleciona, probabilísticamente, um conjunto de programas de computador para serem reproduzidos com base na
    		sua \emph{fitness} (seleção).}
    		\item{Copia alguns dos programas selecionados, sem modificá-los, para a nova população (reprodução).}
    		\item{Cria novos programas de computador por combinar genéticamente partes de dois indivíduos selecionadas 
    		aleatoriamente (cruzamento).}
    		\item{Cria novos programas de computador por substituir partes selecionadas aleatoriamente de um indivíduo por
    		novos indivíduos criados aleatoriamente (mutação).}
    	\end{enumerate}
  	\end{enumerate}
  	\item{Os melhores programas de computador encontrados numa geração são o resultado do processo de \ac{PG} para tal geração.
  	Este resultado pode ser uma solução (ótima ou aproximada) para o problema.}
\end{enumerate}

Nas próximas secções, cada um destes passos é apresentado em mais detalhes.
