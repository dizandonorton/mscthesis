\section{INICIALIZAÇÃO DA POPULAÇÃO}
\label{sec:2inicializacao}

Os indivíduos na população inicial são criados aleatoriamente tal como nos \acp{AG} porém, os métodos utilizados para tal 
são diferentes. Em \ac{PG} existem três métodos comuns de inicialização:

\begin{itemize}
\item{Método completo (\emph{full})}
\item{Método de crescimento (\emph{grow})}
\item{Método em rampa meio-a-meio (\emph{ramped half-and-half})} 
\end{itemize}

No método completo, uma função é selecionada aleatoriamente do conjunto de funções $F$ para ser o nó raiz. De seguida, outras funções são selecionadas 
também aleatoriamente em $F$ para formar os outros nós da árvore. A profundidade\footnote{A profundidade de uma árvore é a quantidade de nós que devem ser percorridos
desde a raiz da árvore até ao nó mais profundo (folha).} máxima da árvore é preenchida apenas por símbolos terminais selecionados 
aleatoriamente no conjunto de terminais $T$. Dessa forma, são criadas árvores completas em que todas as folhas se encontram à mesma 
profundidade. O \algreff{Algoritmo231} contém o pseudocódigo para o método \emph{completo}.

\begin{algorithm}[H]
	\caption{Método de Inicialização Completo}
	\label{Algoritmo231}
	\begin{algorithmic}[1]
		\Function{completo}{$profMaxima$}
			\If {$profMaxima = 1$}
				\State {$no \gets n \in \mathbb{N}, \Call{aridade}{n} = 0$}
			\Else
				\State {$no \gets n \in \mathbb{N}, \Call{aridade}{n} \neq 0$}
				\For {$i \gets 1:\Call{aridade}{n}$}%
					\State {$\Call{addDescendente}{no, \Call{completo}{profMaxima - 1}}$}%
				\EndFor
			\EndIf
			\State \Return {$no$}
		\EndFunction
	\end{algorithmic}
\end{algorithm}

No \algreff{Algoritmo231}, $profMaxima$ é a profundidade máxima da árvore, $\mathbb{N}$ é o conjunto formado pelas funções
em $F$ e os símbolos em $T$, e $n$ é um elemento de $\mathbb{N}$. A função $\Call{aridade}{n}$ determina o número de argumentos
ou operandos de $n$. Quando a aridade de $n$ é igual a zero ($\Call{aridade}{n} = 0$), isto significa que $n$ é um símbolo
terminal. A função $\Call{addDescendente}{no, \Call{completo}{profMaxima-1}}$ conecta o nó-filho $no$ ao seu nó-pai e acrescenta
os outros nós à árvore de forma recursiva.

O método de \emph{crescimento} funciona de forma quase semelhante ao método \emph{completo}, exceto que não são criadas árvores completas 
ou cheias, pois os nós são selecionados aleatoriamente de um conjunto formado pelas funções e os símbolos terminais. Sempre que 
for selecionado um símbolo terminal, o crescimento da árvore para aquele nó termina mesmo que não tenha sido atingida a profundidade 
máxima. O \algreff{Algoritmo232} ilustra o pseudocódigo para o método de \emph{crescimento}.

\begin{algorithm}[H]
	\caption{Método de Inicialização de Crescimento}
	\label{Algoritmo232}
	\begin{algorithmic}[1]
		\Function{crescimento}{$profMaxima$}
			\If {$profMaxima = 1$}
				\State {$no \gets n \in \mathbb{N}, \Call{aridade}{n} = 0$}
			\Else
				\State {$no \gets n \in \mathbb{N}$}
				\For {$i \gets 1:\Call{aridade}{n}$}%
					\State {$\Call{addDescendente}{no, \Call{crescimento}{profMaxima - 1}}$}%
				\EndFor
			\EndIf
			\State \Return {$no$}
		\EndFunction
	\end{algorithmic}
\end{algorithm}

O método \emph{completo} assume a princípio uma estrutura completa para os indivíduos e o método de \emph{crescimento} pode gerar árvores muito 
curtas, caso existam muitos elementos de aridade igual a $0$ (símbolos terminais) \citep{Poli2008}. Numa inicialização em 
\emph{rampa meio-a-meio}, metade da população é criada com o método \emph{completo} e a outra metade é criada com o método de 
\emph{crescimento}. Este procedimento permite a construção de árvores de tamanhos e configurações diferentes, garantindo assim 
a diversidade na população \citep{Koza1992}. O pseudocódigo para o método em \emph{rampa meio-a-meio} é apresentado no \algreff{Algoritmo233}.

\begin{algorithm}[H]
	\caption{Método de Inicialização em Rampa Meio-a-Meio}
	\label{Algoritmo233}
	\begin{algorithmic}[1]
		\Function{rampa}{$profMaxima, probCrescimento$}
			\State {$profundidade \gets \Call{aleatorio}{1,profMaxima}$}%
			\If {$\Call{aleatorio}{0,1} < probCrescimento$}%
				\State \Return {$\Call{crescimento}{profundidade}$}
			\Else
				\State \Return {$\Call{completo}{profundidade}$}
			\EndIf
		\EndFunction
	\end{algorithmic}
\end{algorithm}

No \algreff{Algoritmo233}, $probCrescimento$ é um parâmetro que determina a probabilidade de selecionar o método 
\emph{completo} ou o método
de \emph{crescimento} ao construir uma árvore. A variável $profundidade$ recebe um valor aleatório, entre
$1$ e $profMaxima$, que determina o tamanho para a árvore a ser construída.