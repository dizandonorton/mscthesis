\section{FUNÇÃO DE \emph{FITNESS}}
\label{sec:2funcaofitness}

A capacidade que um indivíduo tem em resolver um problema é quantificada pela função de \emph{fitness}. 
A função de \emph{fitness} avalia a performance do indivíduo executando-o num conjunto de casos de aptidão conhecidos. 
Num problema de regressão simbólica, em que se pretende ajustar uma expressão à um conjunto de dados, os casos de aptidão 
(ou casos de \emph{fitness}) são os valores que as variáveis independentes\footnote{Também conhecida
por: variável de entrada, característica, variável de previsão, atributo, etc.} assumem nos diferentes pontos desse conjunto.

\subsection{\emph{Fitness} bruto}
\label{subsec:2fitnessbruto}

Considerando o valores para $x_1$ e $x_2$ (variáveis independentes) e $y$ (variável dependente\footnote{Também conhecida
por: variável de saída, variável de resposta, valor de saída, etc.}) representados na \tableref{Tabela241}, o trabalho 
da regressão simbólica consistirá em encontrar uma função $f(x_1,x_2)$ que produz valores de saída iguais ou aproximados aos de $y$. 
Uma possível solução é o indivíduo representado pela \figref{Figura222} que origina a função $f(x_1,x_2)= (x_1*x_2 )+(x_1-x_2)$.

\begin{table}[H]
    \begin{tabular}{rrr}%
    \toprule
   	$x_1$ 	&	$x_2$	&	$y$\\ 
   	\midrule
    $1$		&	$-1$	&	$0$\\ 
    $0$		&	$1$		&	$-1$\\ 
   	$-2$	&	$2$		&	$15$\\ 
    $-1$	&	$2$		&	$-8$\\ 
    $2$		&	$-1$	&	$7$\\
    \bottomrule %
    \end{tabular} %
    \centering
    \caption{Casos de \emph{fitness}}
    \label{Tabela241}
\end{table}

Para avaliar a capacidade da função $f(x_1,x_2)$ se ajustar aos dados da \tableref{Tabela241}, deve-se tradicionalmente calcular uma 
medida de erro. Uma medida de erro muito utilizada é o erro quadrático médio (\ac{MSE}), dado pela fórmula:

\begin{equation}
\emph{MSE} = \frac{1}{n}\sum_{i=1}^{n} (y_i-f_i)^2
\label{Equacao241}
\end{equation}
\noindent onde $n$ é o número de casos de \emph{fitness} (linhas, registos ou exemplos), $y_i$ são os valores de saída conhecidos e $f_i$ são os valores 
gerados por \equationref{Equacao221}. Ao aplicar a \equationref{Equacao221} e a \equationref{Equacao241} ao nosso exemplo, obtêm-se os resultados na \tableref{Tabela242}.

\begin{table}[H]
    \begin{tabular}{rrrrr}%
    \toprule
    $x_1$ 	&	$x_2$	&	$y$		&	$f(x_1,x_2)$	& 	$(y-f)^2$\\
    \midrule
    $1$		&	$-1$	&	$0$		&	$1$				&	$1$\\ 
    $0$		&	$1$		&	$-1$	&	$-1$			&	$0$\\ 
   	$-2$	&	$2$		&	$15$	&	$-8$			&	$49$\\ 
    $-1$	&	$2$		&	$-8$	&	$-5$			&	$9$\\ 
    $2$		&	$-1$	&	$7$		&	$1$				&	$36$\\
    		&			&	~		&	\textbf{\emph{MSE}}	&	$95$\\  
    \bottomrule %
    \end{tabular} %
    \centering
    \caption{Valores para $y$, $f$ e erro quadrático médio de $f$}
    \label{Tabela242}
\end{table}

O valor obtido ($95$), é o \emph{fitness} bruto do indivíduo representado pela \figref{Figura222},  para os casos de aptidão na \tableref{Tabela241}. 
O \emph{fitness} bruto expressa a aptidão da solução numa terminologia natural do problema \citep{Koza1992}.

\subsection{\emph{Fitness} padronizado}

O \emph{fitness} padronizado é uma transformação ao \emph{fitness} bruto de formas a que um valor menor de \emph{fitness}, é considerado melhor.
Em muitos casos é conveniente e desejável fazer com que o melhor valor de \emph{fitness} padronizado seja igual a zero. Isto pode ser obtido através da
soma ou subtração de uma constante. Num problema em que um valor de \emph{fitness} bruto é melhor e o valor máximo de \emph{fitness} bruto é conhecido,
o \emph{fitness} padronizado pode ser obtido pela fórmula:

\begin{equation}
f_S(i)=f^{max}_R - f_R(i)
\label{Equacao242}
\end{equation}
\noindent onde $f_R(i)$ é o \emph{fitness} bruto de $i$.