\subsection{Fludarabina}
\label{subsec:61fludarabina}

A \figref{Figura621fludarabine} ilustra a mediana do \emph{fitness} de teste dos melhores indivíduos num total de $30$ execuções independentes.
Pode-se verificar que as versões RMSE-GP, MASE-GP, R2-GP e B-GP apresentam as melhores medianas ao longo das $200$ gerações. Por sua vez, 
as versões LS-GP e PCCN-GP apresentam uma mediana que piora ao longo das gerações e apresentam as piores medianas comparando com 
as outras versões de \ac{PG}, exceto o NS-GP ilustrado na \figref{Figura622fludarabine}.

\begin{figure}[H]
	\centering
	\begin{tikzpicture}[baseline]
	\begin{axis}[
		xlabel={Gerações}, 
		ylabel={\emph{Fitness} de teste},
		minor y tick num=1, 
		xmin=0, 
		xmax = 200, 
		smooth, 
		no markers,
		legend style={
			font=\small,
			line width=1pt,
			cells={anchor=east},
			legend pos=outer north east,
			legend cell align = left
		},
		]
		\addplot[color=blue, very thick] table [x expr=\coordindex, y expr=\thisrowno{0}, ]  {results/fludarabine/rmse/medianTestFitness.dat};
		\addplot[color=red, very thick] table [x expr=\coordindex, y expr=\thisrowno{0}, ]  {results/fludarabine/pcc/medianTestFitness.dat};
		\addplot[color=green, very thick] table [x expr=\coordindex, y expr=\thisrowno{0}, ]  {results/fludarabine/ls/medianTestFitness.dat};
		\addplot[color=brown, very thick] table [x expr=\coordindex, y expr=\thisrowno{0}, ]  {results/fludarabine/mase/medianTestFitness.dat};
		\addplot[color=black, very thick] table [x expr=\coordindex, y expr=\thisrowno{0}, ]  {results/fludarabine/rs/medianTestFitness.dat};
		\addplot[color=orange, very thick] table [x expr=\coordindex, y expr=\thisrowno{0}, ]  {results/fludarabine/gpb/medianTestFitness.dat};
	\legend{RMSE-GP, PCCN-GP, LS-GP, MASE-GP, R2-GP, B-GP}
	\end{axis}
	\end{tikzpicture}
	\caption{Mediana do \emph{fitness} de teste dos melhores indivíduos encontrados no conjunto de treino}
	\label{Figura621fludarabine}
\end{figure}

Na \figref{Figura622fludarabine} são ilustrados a parte os resultados para o NS-GP, devido a enorme diferença de escala (na ordem dos
milhares) comparando com as versões anteriores (na ordem das décimas). Nesta figura, os resultados da mediana no \emph{fitness} de 
teste decresce rapidamente nas primeiras gerações, e depois mantém-se constante até a última geração.

\begin{figure}[H]
	\centering
	\begin{tikzpicture}[baseline]
	\begin{axis}[
		xlabel={Gerações}, 
		ylabel={\emph{Fitness} de teste},
		minor y tick num=1, 
		xmin=0, 
		xmax = 200, 
		smooth, 
		no markers,
		legend style={
			font=\small,
			line width=1pt,
			cells={anchor=east},
			legend pos=outer north east,
			legend cell align = left
		},
		]
		\addplot [color=blue, very thick] table [x expr=\coordindex, y expr=\thisrowno{0}, ]  {results/fludarabine/ns/medianTestFitness.dat};
	\legend{NS-GP}
	\end{axis}
	\end{tikzpicture}
	\caption{Mediana do \emph{fitness} de teste dos melhores indivíduos encontrados no conjunto de treino (NS-GP)}
	\label{Figura622fludarabine}
\end{figure}

A \tableref{Tabela621fludarabine} apresenta o \emph{fitness} de teste, a média do \emph{fitness} de teste e o desvio padrão do 
\emph{fitness} do melhor indivíduo encontrado ao longo das $200$ gerações em $30$ execuções independentes no conjunto de treino para
todas versões de \ac{PG} utilizadas. O melhor valor do \emph{fitness} de teste foi retornado por MASE-GP, sendo tal valor muito aproximado
aos retornados por R2-GP, RMSE-GP, B-GP e NS-GP. De forma geral, estas versões de \ac{PG} apresentaram também os melhores valores
do \emph{fitness} médio e do desvio padrão. É de notar que o \emph{fitness} de teste retornado por NS-GP é melhor que os retornados 
por LS-GP e PCCN-GP.

\begin{table}[H]
	\begin{tabular}{lrrr}%
	\toprule	
	&	\textbf{\emph{Fitness}}	&	\textbf{\emph{Fitness} médio}	& \textbf{Desvio padrão} \\ 
	\midrule %
	RMSE-GP			&	$\input{results/fludarabine/rmse/bestTest.dat}$	& $\input{results/fludarabine/rmse/avgTestFitness.dat}$ & $\input{results/fludarabine/rmse/stdTestFitness.dat}$ \\
   	LS-GP			&	$\input{results/fludarabine/ls/bestTest.dat}$	& $\input{results/fludarabine/ls/avgTestFitness.dat}$ & $\input{results/fludarabine/ls/stdTestFitness.dat}$ \\
	PCCN-GP			&	$\input{results/fludarabine/pcc/bestTest.dat}$	& $\input{results/fludarabine/pcc/avgTestFitness.dat}$ & $\input{results/fludarabine/pcc/stdTestFitness.dat}$  \\
   	MASE-GP			&	$\input{results/fludarabine/mase/bestTest.dat}$	& $\input{results/fludarabine/mase/avgTestFitness.dat}$ & $\input{results/fludarabine/mase/stdTestFitness.dat}$  \\
   	R2-GP			&	$\input{results/fludarabine/rs/bestTest.dat}$	& $\input{results/fludarabine/rs/avgTestFitness.dat}$ & $\input{results/fludarabine/rs/stdTestFitness.dat}$ \\
   	B-GP			&	$\input{results/fludarabine/gpb/bestTest.dat}$	& $\input{results/fludarabine/gpb/avgTestFitness.dat}$ & $\input{results/fludarabine/gpb/stdTestFitness.dat}$  \\
   	NS-GP			&	$\input{results/fludarabine/ns/bestTest.dat}$	& $\input{results/fludarabine/ns/avgTestFitness.dat}$ & $\input{results/fludarabine/ns/stdTestFitness.dat}$  \\
	\bottomrule %
	\end{tabular}%
	\centering
	\caption{Melhor \emph{fitness} de teste, média do \emph{fitness} de teste e desvio padrão do \emph{fitness} de teste}
	\label{Tabela621fludarabine}
\end{table}

A \figref{Figura623fludarabine} reporta a mediana da quantidade de nós (ou tamanho) dos melhores indivíduos
encontrados no conjunto de treino ao longo das $200$ nas $30$ execuções independentes. O tamanho dos melhores indivíduos encontrados por RMSE-GP, MASE-GP, 
R2-GP, B-GP e NS-GP são muito semelhantes e mantêm-se relativamente constantes durante as $200$ gerações.
Diferentemente das outras versões de PG, a mediana do tamanho das melhores soluções encontradas pelo PCCN-GP e
LS-GP aumenta progressivamente ao longo das  $200$ gerações, em $30$ execuções independentes.


\begin{figure}[H]
	\centering
	\begin{tikzpicture}[baseline]
	\begin{axis}[
		xlabel={Gerações}, 
		ylabel={Tamanho},
		minor y tick num=1, 
		xmin=0, 
		xmax = 200, 
		smooth, 
		no markers,
		legend style={
			font=\small,
			line width=1pt,
			cells={anchor=east},
			legend pos=outer north east,
			legend cell align = left
		},
		]
		\addplot[color=blue, very thick] table [x expr=\coordindex, y expr=\thisrowno{0}, ]  {results/fludarabine/rmse/medianNodes.dat};
		\addplot[color=red, very thick] table [x expr=\coordindex, y expr=\thisrowno{0}, ]  {results/fludarabine/pcc/medianNodes.dat};
		\addplot[color=green, very thick] table [x expr=\coordindex, y expr=\thisrowno{0}, ]  {results/fludarabine/ls/medianNodes.dat};
		\addplot[color=brown, very thick] table [x expr=\coordindex, y expr=\thisrowno{0}, ]  {results/fludarabine/mase/medianNodes.dat};
		\addplot[color=black, very thick] table [x expr=\coordindex, y expr=\thisrowno{0}, ]  {results/fludarabine/rs/medianNodes.dat};
		\addplot[color=orange, very thick] table [x expr=\coordindex, y expr=\thisrowno{0}, ]  {results/fludarabine/gpb/medianNodes.dat};
		\addplot[color=gray, very thick] table [x expr=\coordindex, y expr=\thisrowno{0}, ]  {results/fludarabine/ns/medianNodes.dat};
	\legend{RMSE-GP, PCCN-GP, LS-GP, MASE-GP, R2-GP, B-GP, NS-GP}
	\end{axis}
	\end{tikzpicture}
	\caption{Mediana do tamanho (nº de nós) do melhor indivíduo no conjunto de treino}
	\label{Figura623fludarabine}
\end{figure}

%PERCENTAGENS

De seguida, reportamos a percentagem de ocorrência das $10$ variáveis mais utilizadas pelos melhores indivíduos nas $30$ execuções
para as diferentes versões de PG, sobre o conjunto de dados da Fludarabina. Pelas tabelas \ref{Tabela622fludarabine}, 
\ref{Tabela623fludarabine} e \ref{Tabela624fludarabine} pode-se notar a presença repetida das variáveis $x_{520}$, $x_{629}$ 
e $x_{283}$ que são utilizadas por R2-GP e RMSE-GP, que apresentam os melhores resultados de \emph{fitness} sobre o conjunto de teste.

\pgfplotstableread[col sep=comma]{results/fludarabine/gpb/featuresCountTable.txt}\fctgpb
\pgfplotstableread[col sep=comma]{results/fludarabine/ls/featuresCountTable.txt}\fctls
\pgfplotstableread[col sep=comma]{results/fludarabine/mase/featuresCountTable.txt}\fctmase
\pgfplotstableread[col sep=comma]{results/fludarabine/ns/featuresCountTable.txt}\fctns
\pgfplotstableread[col sep=comma]{results/fludarabine/pcc/featuresCountTable.txt}\fctpcc
\pgfplotstableread[col sep=comma]{results/fludarabine/rmse/featuresCountTable.txt}\fctrmse
\pgfplotstableread[col sep=comma]{results/fludarabine/rs/featuresCountTable.txt}\fctrs

%gpb, ls, mase
\begin{table}[H]\small\centering
	\begin{minipage}{.3\linewidth}
		\centering
		\pgfplotstabletypeset[
			every head row/.style={
				before row={%
					\multicolumn{2}{c}{B-GP}\\
					\toprule
				},
				after row=\midrule,
			},
			every last row/.style={after row=\bottomrule},
			columns = {variable,pct},
			columns/pct/.style = {column name = $\%$, column type = r},
			columns/variable/.style = {
				column name = {\bf Variável},
				string type,
				column type = c,
				postproc cell content/.append style={/pgfplots/table/@cell content/.add={$}{$},},
			},
			%every head row/.style={before row=\toprule,after row=\midrule},
		]\fctgpb
		
	\end{minipage}\hspace*{1em}
	\begin{minipage}{.3\linewidth}
		\centering
		\pgfplotstabletypeset[
			every head row/.style={
				before row={%
					\multicolumn{2}{c}{LS-GP}\\
					\toprule
				},
				after row=\midrule,
			},
			every last row/.style={after row=\bottomrule},
			columns = {variable,pct},
			columns/pct/.style = {column name = $\%$, column type = r},
			columns/variable/.style = {
				column name = {\bf Variável},
				string type,
				column type = c,
				postproc cell content/.append style={/pgfplots/table/@cell content/.add={$}{$},},
			},
			%every head row/.style={before row=\toprule,after row=\midrule},
		]\fctls
		
	\end{minipage}\hspace*{1em}
	\begin{minipage}{.3\linewidth}
		\centering
		\pgfplotstabletypeset[
			every head row/.style={
				before row={%
					\multicolumn{2}{c}{MASE-GP}\\
					\toprule
				},
				after row=\midrule,
			},
			every last row/.style={after row=\bottomrule},
			columns = {variable,pct},
			columns/pct/.style = {column name = $\%$, column type = r},
			columns/variable/.style = {
				column name = {\bf Variável},
				string type,
				column type = c,
				postproc cell content/.append style={/pgfplots/table/@cell content/.add={$}{$},},
			},
			%every head row/.style={before row=\toprule,after row=\midrule},
		]\fctmase
	\end{minipage}
	\caption{\small{Percentagem de ocorrência das 10 variáveis mais utilizadas pelos melhores indivíduos}}
	\label{Tabela622fludarabine}
\end{table}
%
\vspace*{0.5mm}
%gpb, ls, mase
\begin{table}[H]\small\centering
	\begin{minipage}{.3\linewidth}
		\centering
		\pgfplotstabletypeset[
			every head row/.style={
				before row={%
					\multicolumn{2}{c}{PCCN-GP}\\
					\toprule
				},
				after row=\midrule,
			},
			every last row/.style={after row=\bottomrule},
			columns = {variable,pct},
			columns/pct/.style = {column name = $\%$, column type = r},
			columns/variable/.style = {
				column name = {\bf Variável},
				string type,
				column type = c,
				postproc cell content/.append style={/pgfplots/table/@cell content/.add={$}{$},},
			},
			%every head row/.style={before row=\toprule,after row=\midrule},
		]\fctpcc
		
	\end{minipage}\hspace*{1em}
	\begin{minipage}{.3\linewidth}
		\centering
		\pgfplotstabletypeset[
			every head row/.style={
				before row={%
					\multicolumn{2}{c}{RMSE-GP}\\
					\toprule
				},
				after row=\midrule,
			},
			every last row/.style={after row=\bottomrule},
			columns = {variable,pct},
			columns/pct/.style = {column name = $\%$, column type = r},
			columns/variable/.style = {
				column name = {\bf Variável},
				string type,
				column type = c,
				postproc cell content/.append style={/pgfplots/table/@cell content/.add={$}{$},},
			},
			%every head row/.style={before row=\toprule,after row=\midrule},
		]\fctrmse
		
	\end{minipage}\hspace*{1em}
	\begin{minipage}{.3\linewidth}
		\centering
		\pgfplotstabletypeset[
			every head row/.style={
				before row={%
					\multicolumn{2}{c}{R2-GP}\\
					\toprule
				},
				after row=\midrule,
			},
			every last row/.style={after row=\bottomrule},
			columns = {variable,pct},
			columns/pct/.style = {column name = $\%$, column type = r},
			columns/variable/.style = {
				column name = {\bf Variável},
				string type,
				column type = c,
				postproc cell content/.append style={/pgfplots/table/@cell content/.add={$}{$},},
			},
			%every head row/.style={before row=\toprule,after row=\midrule},
		]\fctrs
	\end{minipage}
	\caption{\small{Percentagem de ocorrência das 10 variáveis mais utilizadas pelos melhores indivíduos}}
	\label{Tabela623fludarabine}
\end{table}
%NS
\vspace*{0.5mm}
%gpb, ls, mase
\begin{table}[H]\small\centering
	\begin{minipage}{1\linewidth}
		\centering
		\pgfplotstabletypeset[
			every head row/.style={
				before row={%
					\multicolumn{2}{c}{NS-GP}\\
					\toprule
				},
				after row=\midrule,
			},
			every last row/.style={after row=\bottomrule},
			columns = {variable,pct},
			columns/pct/.style = {column name = $\%$, column type = r},
			columns/variable/.style = {
				column name = {\bf Variável},
				string type,
				column type = c,
				postproc cell content/.append style={/pgfplots/table/@cell content/.add={$}{$},},
			},
			%every head row/.style={before row=\toprule,after row=\midrule},
		]\fctns		
	\end{minipage}
	\caption{\small{Percentagem de ocorrência das 10 variáveis mais utilizadas pelos melhores indivíduos}}
	\label{Tabela624fludarabine}
\end{table}