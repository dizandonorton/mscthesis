\subsection{\ac{LD50}}
\label{subsec:61ld50}


A \figref{Figura621ld50} ilustra a mediana do \emph{fitness} de teste dos melhores indivíduos, num total de $30$ execuções independentes.
Pode-se verificar que as versões RMSE-GP, MASE-GP e R2-GP apresentam as melhores medianas ao longo das $200$ gerações. 
O PCCN-GP e LS-GP apresentam as piores medianas do \emph{fitness} de teste e é de notar que se mantêm ligeiramente constantes
durante as $200$ gerações. Por sua vez, a versão B-GP apresenta uma mediana que oscila bastante nas primeiras $100$ gerações
e mantêm-se sútilmente constante até a última geração.


\begin{figure}[H]
	\centering
	\begin{tikzpicture}[baseline]
	\begin{axis}[
		xlabel={Gerações}, 
		ylabel={\emph{Fitness} de teste},
		minor y tick num=1, 
		xmin=0, 
		xmax = 200, 
		smooth, 
		no markers,
		legend style={
			font=\small,
			line width=1pt,
			cells={anchor=east},
			legend pos=outer north east,
			legend cell align = left
		},
		]
		\addplot[color=blue, very thick] table [x expr=\coordindex, y expr=\thisrowno{0}, ]  {results/ld50/rmse/medianTestFitness.dat};
		\addplot[color=red, very thick] table [x expr=\coordindex, y expr=\thisrowno{0}, ]  {results/ld50/pcc/medianTestFitness.dat};
		\addplot[color=green, very thick] table [x expr=\coordindex, y expr=\thisrowno{0}, ]  {results/ld50/ls/medianTestFitness.dat};
		\addplot[color=brown, very thick] table [x expr=\coordindex, y expr=\thisrowno{0}, ]  {results/ld50/mase/medianTestFitness.dat};
		\addplot[color=black, very thick] table [x expr=\coordindex, y expr=\thisrowno{0}, ]  {results/ld50/rs/medianTestFitness.dat};
		\addplot[color=orange, very thick] table [x expr=\coordindex, y expr=\thisrowno{0}, ]  {results/ld50/gpb/medianTestFitness.dat};
	\legend{RMSE-GP, PCCN-GP, LS-GP, MASE-GP, R2-GP, B-GP}
	\end{axis}
	\end{tikzpicture}
	\caption{Mediana do \emph{fitness} de teste dos melhores indivíduos encontrados no conjunto de treino}
	\label{Figura621ld50}
\end{figure}

Na \figref{Figura622ld50}, são ilustrados à parte os resultados para o NS-GP devido a enorme diferença de escala (na ordem de $10^{10}$)
comparando com as versões anteriores (na ordem das unidades de milhar). Nesta figura, os resultados da mediana no \emph{fitness} de 
teste apresentam um comportamento esquisito pois decrescem rapidamente na geração inicial, e de seguida mantém-se constante até a 
última geração.

\begin{figure}[H]
	\centering
	\begin{tikzpicture}[baseline]
	\begin{axis}[
		xlabel={Gerações}, 
		ylabel={\emph{Fitness} de teste},
		minor y tick num=1, 
		xmin=0, 
		xmax = 200, 
		smooth, 
		no markers,
		legend style={
			font=\small,
			line width=1pt,
			cells={anchor=east},
			legend pos=outer north east,
			legend cell align = left
		},
		]
		\addplot [color=blue, very thick] table [x expr=\coordindex, y expr=\thisrowno{0}, ]  {results/ld50/ns/medianTestFitness.dat};
	\legend{NS-GP}
	\end{axis}
	\end{tikzpicture}
	\caption{Mediana do \emph{fitness} de teste dos melhores indivíduos encontrados no conjunto de treino (NS-GP)}
	\label{Figura622ld50}
\end{figure}

Na \tableref{Tabela621ld50} são ilustrados o \emph{fitness} de teste, a média do \emph{fitness} de teste e o desvio padrão do 
\emph{fitness} do melhor indivíduo encontrado ao longo das $200$ gerações em $30$ execuções independentes no conjunto de treino. 

O melhor \emph{fitness} de teste é apresentado pelo melhor indivíduo do R2-GP, seguido por MASE-GP e B-GP. Por outro lado, 
o RMSE-GP apresenta os melhores \emph{fitness} médio e desvio padrão comparando com as outras versões. Curiosamente, o \emph{fitness}
encontrando por NS-GP é melhor que o de LS-GP. O PCCN-GP apresentou os piores resultados e isto indica a fraca correlação linear
entre as soluções encontradas e as soluções reais.

\begin{table}[H]
	\begin{tabular}{lrrr}%
	\toprule	
	&	\textbf{\emph{Fitness}}	&	\textbf{\emph{Fitness} médio}	& \textbf{Desvio padrão} \\ 
	\midrule %
	RMSE-GP			&	$\input{results/ld50/rmse/bestTest.dat}$	& $\input{results/ld50/rmse/avgTestFitness.dat}$ & $\input{results/ld50/rmse/stdTestFitness.dat}$ \\
   	LS-GP			&	$\input{results/ld50/ls/bestTest.dat}$	& $\input{results/ld50/ls/avgTestFitness.dat}$ & $\input{results/ld50/ls/stdTestFitness.dat}$ \\
	PCCN-GP			&	$\input{results/ld50/pcc/bestTest.dat}$	& $\input{results/ld50/pcc/avgTestFitness.dat}$ & $\input{results/ld50/pcc/stdTestFitness.dat}$  \\
   	MASE-GP			&	$\input{results/ld50/mase/bestTest.dat}$	& $\input{results/ld50/mase/avgTestFitness.dat}$ & $\input{results/ld50/mase/stdTestFitness.dat}$  \\
   	R2-GP			&	$\input{results/ld50/rs/bestTest.dat}$	& $\input{results/ld50/rs/avgTestFitness.dat}$ & $\input{results/ld50/rs/stdTestFitness.dat}$ \\
   	B-GP			&	$\input{results/ld50/gpb/bestTest.dat}$	& $\input{results/ld50/gpb/avgTestFitness.dat}$ & $\input{results/ld50/gpb/stdTestFitness.dat}$  \\
   	NS-GP			&	$\input{results/ld50/ns/bestTest.dat}$	& $\input{results/ld50/ns/avgTestFitness.dat}$ & $\input{results/ld50/ns/stdTestFitness.dat}$  \\
	\bottomrule %
	\end{tabular}%
	\centering
	\caption{Melhor \emph{fitness} de teste, média do \emph{fitness} de teste e desvio padrão do \emph{fitness} de teste}
	\label{Tabela621ld50}
\end{table}

A \figref{Figura623ld50} reporta a mediana do tamanho do melhores indivíduos
encontrados no conjunto de treino ao longo das $200$ nas $30$ execuções independentes. O tamanho dos melhores indivíduos encontrados por RMSE-GP, PCCN-GP, LS-GP, MASE-GP e R2-GP
varia entre $60$ à $100$ nós. O B-GP e o NS-GP apresentam um comportamento diferente das outras versões de PG. Para o B-GP, 
a mediana do tamanho dos indivíduos cresce até aproximadamente a $50\textsuperscript{a}$ geração, e permanece relativamente constante (entre
$20$ à $25$ nós) até a última geração. Quanto ao NS-GP, os valores decrescem rápidamente nas gerações iniciais e permanecem 
constantes até à última geração.

\begin{figure}[H]
	\centering
	\begin{tikzpicture}[baseline]
	\begin{axis}[
		xlabel={Gerações}, 
		ylabel={Tamanho},
		minor y tick num=1, 
		xmin=0, 
		xmax = 200, 
		smooth, 
		no markers,
		legend style={
			font=\small,
			line width=1pt,
			cells={anchor=east},
			legend pos=outer north east,
			legend cell align = left
		},
		]
		\addplot[color=blue, very thick] table [x expr=\coordindex, y expr=\thisrowno{0}, ]  {results/ld50/rmse/medianNodes.dat};
		\addplot[color=red, very thick] table [x expr=\coordindex, y expr=\thisrowno{0}, ]  {results/ld50/pcc/medianNodes.dat};
		\addplot[color=green, very thick] table [x expr=\coordindex, y expr=\thisrowno{0}, ]  {results/ld50/ls/medianNodes.dat};
		\addplot[color=brown, very thick] table [x expr=\coordindex, y expr=\thisrowno{0}, ]  {results/ld50/mase/medianNodes.dat};
		\addplot[color=black, very thick] table [x expr=\coordindex, y expr=\thisrowno{0}, ]  {results/ld50/rs/medianNodes.dat};
		\addplot[color=orange, very thick] table [x expr=\coordindex, y expr=\thisrowno{0}, ]  {results/ld50/gpb/medianNodes.dat};
		\addplot[color=gray, very thick] table [x expr=\coordindex, y expr=\thisrowno{0}, ]  {results/ld50/ns/medianNodes.dat};
	\legend{RMSE-GP, PCCN-GP, LS-GP, MASE-GP, R2-GP, B-GP, NS-GP}
	\end{axis}
	\end{tikzpicture}
	\caption{Mediana do tamanho (nº de nós) do melhor indivíduo no conjunto de treino}
	\label{Figura623ld50}
\end{figure}


%PERCENTAGENS
Finalmente, reportamos a percentagem de ocorrência das $10$ variáveis mais utilizadas pelos melhores indivíduos nas $30$ execuções
para as diferentes versões de PG, sobre o conjunto de dados da \ac{LD50}. Pelas tabelas \ref{Tabela622ld50}, \ref{Tabela623ld50}
e \ref{Tabela624ld50} pode-se notar a presença repetida das variáveis $x_{11}$ (em LS-GP, MASE-GP, PCCN-GP), $x_{151}$ (em MASE-GP, 
RMSE-GP, NS-GP) e $x_{51}$ (em MASE-GP, RMSE-GP, R2-GP).

\pgfplotstableread[col sep=comma]{results/ld50/gpb/featuresCountTable.txt}\fctgpb
\pgfplotstableread[col sep=comma]{results/ld50/ls/featuresCountTable.txt}\fctls
\pgfplotstableread[col sep=comma]{results/ld50/mase/featuresCountTable.txt}\fctmase
\pgfplotstableread[col sep=comma]{results/ld50/ns/featuresCountTable.txt}\fctns
\pgfplotstableread[col sep=comma]{results/ld50/pcc/featuresCountTable.txt}\fctpcc
\pgfplotstableread[col sep=comma]{results/ld50/rmse/featuresCountTable.txt}\fctrmse
\pgfplotstableread[col sep=comma]{results/ld50/rs/featuresCountTable.txt}\fctrs

%gpb, ls, mase
\begin{table}[H]\small\centering
	\begin{minipage}{.3\linewidth}
		\centering
		\pgfplotstabletypeset[
			every head row/.style={
				before row={%
					\multicolumn{2}{c}{B-GP}\\
					\toprule
				},
				after row=\midrule,
			},
			every last row/.style={after row=\bottomrule},
			columns = {variable,pct},
			columns/pct/.style = {column name = $\%$, column type = r},
			columns/variable/.style = {
				column name = {\bf Variável},
				string type,
				column type = c,
				postproc cell content/.append style={/pgfplots/table/@cell content/.add={$}{$},},
			},
			%every head row/.style={before row=\toprule,after row=\midrule},
		]\fctgpb
		
	\end{minipage}\hspace*{1em}
	\begin{minipage}{.3\linewidth}
		\centering
		\pgfplotstabletypeset[
			every head row/.style={
				before row={%
					\multicolumn{2}{c}{LS-GP}\\
					\toprule
				},
				after row=\midrule,
			},
			every last row/.style={after row=\bottomrule},
			columns = {variable,pct},
			columns/pct/.style = {column name = $\%$, column type = r},
			columns/variable/.style = {
				column name = {\bf Variável},
				string type,
				column type = c,
				postproc cell content/.append style={/pgfplots/table/@cell content/.add={$}{$},},
			},
			%every head row/.style={before row=\toprule,after row=\midrule},
		]\fctls
		
	\end{minipage}\hspace*{1em}
	\begin{minipage}{.3\linewidth}
		\centering
		\pgfplotstabletypeset[
			every head row/.style={
				before row={%
					\multicolumn{2}{c}{MASE-GP}\\
					\toprule
				},
				after row=\midrule,
			},
			every last row/.style={after row=\bottomrule},
			columns = {variable,pct},
			columns/pct/.style = {column name = $\%$, column type = r},
			columns/variable/.style = {
				column name = {\bf Variável},
				string type,
				column type = c,
				postproc cell content/.append style={/pgfplots/table/@cell content/.add={$}{$},},
			},
			%every head row/.style={before row=\toprule,after row=\midrule},
		]\fctmase
	\end{minipage}
	\caption{\small{Percentagem de ocorrência das 10 variáveis mais utilizadas pelos melhores indivíduos}}
	\label{Tabela622ld50}
\end{table}
%
\vspace*{0.5mm}
%gpb, ls, mase
\begin{table}[H]\small\centering
	\begin{minipage}{.3\linewidth}
		\centering
		\pgfplotstabletypeset[
			every head row/.style={
				before row={%
					\multicolumn{2}{c}{PCCN-GP}\\
					\toprule
				},
				after row=\midrule,
			},
			every last row/.style={after row=\bottomrule},
			columns = {variable,pct},
			columns/pct/.style = {column name = $\%$, column type = r},
			columns/variable/.style = {
				column name = {\bf Variável},
				string type,
				column type = c,
				postproc cell content/.append style={/pgfplots/table/@cell content/.add={$}{$},},
			},
			%every head row/.style={before row=\toprule,after row=\midrule},
		]\fctpcc
		
	\end{minipage}\hspace*{1em}
	\begin{minipage}{.3\linewidth}
		\centering
		\pgfplotstabletypeset[
			every head row/.style={
				before row={%
					\multicolumn{2}{c}{RMSE-GP}\\
					\toprule
				},
				after row=\midrule,
			},
			every last row/.style={after row=\bottomrule},
			columns = {variable,pct},
			columns/pct/.style = {column name = $\%$, column type = r},
			columns/variable/.style = {
				column name = {\bf Variável},
				string type,
				column type = c,
				postproc cell content/.append style={/pgfplots/table/@cell content/.add={$}{$},},
			},
			%every head row/.style={before row=\toprule,after row=\midrule},
		]\fctrmse
		
	\end{minipage}\hspace*{1em}
	\begin{minipage}{.3\linewidth}
		\centering
		\pgfplotstabletypeset[
			every head row/.style={
				before row={%
					\multicolumn{2}{c}{R2-GP}\\
					\toprule
				},
				after row=\midrule,
			},
			every last row/.style={after row=\bottomrule},
			columns = {variable,pct},
			columns/pct/.style = {column name = $\%$, column type = r},
			columns/variable/.style = {
				column name = {\bf Variável},
				string type,
				column type = c,
				postproc cell content/.append style={/pgfplots/table/@cell content/.add={$}{$},},
			},
			%every head row/.style={before row=\toprule,after row=\midrule},
		]\fctrs
	\end{minipage}
	\caption{\small{Percentagem de ocorrência das 10 variáveis mais utilizadas pelos melhores indivíduos}}
	\label{Tabela623ld50}
\end{table}
%NS
\vspace*{0.5mm}
%gpb, ls, mase
\begin{table}[H]\small\centering
	\begin{minipage}{1\linewidth}
		\centering
		\pgfplotstabletypeset[
			every head row/.style={
				before row={%
					\multicolumn{2}{c}{NS-GP}\\
					\toprule
				},
				after row=\midrule,
			},
			every last row/.style={after row=\bottomrule},
			columns = {variable,pct},
			columns/pct/.style = {column name = $\%$, column type = r},
			columns/variable/.style = {
				column name = {\bf Variável},
				string type,
				column type = c,
				postproc cell content/.append style={/pgfplots/table/@cell content/.add={$}{$},},
			},
			%every head row/.style={before row=\toprule,after row=\midrule},
		]\fctns		
	\end{minipage}
	\caption{\small{Percentagem de ocorrência das 10 variáveis mais utilizadas pelos melhores indivíduos}}
	\label{Tabela624ld50}
\end{table}