\section{INTRODUÇÃO À PROGRAMAÇÃO GENÉTICA}
\label{sec:introgp}

John Koza introduziu o conceito de \ac{PG} onde os indivíduos são representados por programas de computador 
(e.g: expressões matemáticas, expressões lógicas, etc.). Desta forma, em vez de a pesquisa ser feita num conjunto de soluções, 
são gerados programas que automaticamente resolvem o problema sem a necessidade de se saber à partida a estrutura da solução 
\citep{Koza1992}.

Uma forma comum de representar os programas de computador em \ac{PG} é por intermédio de árvores de sintaxe que 
podem ser facilmente transformadas em programas nas linguagens de programação conhecidas. Na implementação
original da \ac{PG}, o \ac{LISP} foi a linguagem de programação utilizada \citep{Koza1992}.

Os indivíduos da população inicial são criados aleatoriamente e ao longo das gerações são produzidos novos indivíduos pela 
aplicação dos operadores genéticos. Os indivíduos mais aptos (com maior valor de \emph{fitness} em problemas de maximização 
ou com menor valor de \emph{fitness} em problemas de minimização) são copiados para as novas gerações ou selecionados para 
cruzamento ou mutação. 

Tal como acontece com os \acp{AG}, as operações de seleção, reprodução, cruzamento e mutação também se aplicam 
ao indivíduos em \ac{PG} (que são programas de computador). Estes e outros aspetos mais avançados sobre a 
\ac{PG} tradicional são o assunto do capítulo \ref{cap:2}.