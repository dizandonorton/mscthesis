\section{ESTRUTURA}
\label{sec:estrutura}

Para além do presente capítulo de introdução, este trabalho de projeto está dividido pelos seguintes outros capítulos:

\begin{itemize}
\item{O capítulo \ref{cap:2} fornece uma introdução a \ac{PG} padrão, %e o seu estado de arte, 
apresentado os principais conceitos}
\item{O capítulo \ref{cap:3} descreve a importância da previsão dos parâmetros farmacocinéticos bem como os principais 
resultados encontrados na literatura}
%\item{Os Sistemas de Recomendação Musical (SRM), suas vantagens e aplicações são brevemente apresentadas no capítulo \ref{cap:4}. No final deste capítulo é apresenta a tarefa de SRM tratada no presente trabalho: previsão do ano de lançamento de músicas.}
\item{O capítulo \ref{cap:5} apresenta a metodologia, técnicas, configurações utilizadas para a recolha e análise dos dados, configuração do ambiente de execução dos testes e a ferramenta utilizada durante o processo de previsão
% (GPLAB). 
%No final do capítulo são apresentadas as extensões feitas ao GPLAB.
}
\item{Os resultados do presente trabalho são apresentados e discutidos no capítulo \ref{cap:6} 
%Posteriormente é elaborada uma comparação com os principais resultados obtidos por outros algoritmos de previsão e com os resultados publicados até ao momento.
}
\item{Finalmente, o capítulo \ref{cap:finais} conclui o presente relatório e traça o caminho para a elaboração de trabalhos futuros
%São também apresentadas as limitações do trabalho.
}
%\item{O código-fonte implementado, as extensões ao GPLAB e outros gráficos são apresentados no Apêndice \ref{ap:a}.}
\end{itemize}