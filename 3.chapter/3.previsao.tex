\section{PREVISÃO DE PARÂMETROS FARMACOCINÉTICOS}
\label{sec:3previsao}

As ferramentas de simulação computacional (\emph{in silico}) para a previsão de parâmetros farmacocinéticos são de particular interesse na
indústria farmacêutica. Para além de pouparem tempo e recursos, os modelos computacionais permitem também que moléculas inapropriadas
sejam descartadas durante a fase inicial do processo de desenvolvimento de fármacos. O objetivo dos modelos \emph{in silico} de absorção, distribuição,
metabolismo e excreção é obter uma previsão \emph{in vivo}, o mais precisa possível, do efeito das potenciais molêculas de um medicamento no 
organismo humano \citep{waterbeemd2003,madan2012prediction}. Existem básicamente dois métodos de modelação computacional: 
modelação molecular e modelação de dados. Os métodos de modelação molecular 
utilizam cálculos intensivos nas estruturas proteícas. Os métodos baseados em modelação de dados são amplamente divulgados na literatura e 
pertencem a categoria de modelos de previsão chamados de ``modelos da Relação Quantitativa Estrutura-Atividade'' (\ac{QSAR}).

Os modelos \ac{QSAR} têm por objetivo definir uma relação quantitativa entre a estrutura de uma molécula e suas atividades biológicas.
Para tal, é necessário obter um conjunto de dados de treino de fármacos para os quais são conhecidos os parâmetros de atividade biológica.
Um enorme conjunto de características, ou descritores moleculares, são calculados apartir da estrutura molecular de cada composto. A construção de
modelos \ac{QSAR} geralmente envolve a execução dos seguintes três passos:
\begin{enumerate}
  \item {Adquirir, ou se possível, desenvolver um conjunto de dados de treino de compostos químicos para os quais se conhecem os 
  parâmetros biológicos}
  \item {Obter os descritores moleculares que relacionem-se adequadamente com a atividade biológica}
  \item {Aplicar métodos para construir uma relação matemática que permite calcular a atividade biológica}
\end{enumerate}

A obtenção de modelos \ac{QSAR} de qualidade, capazes de prever a atividade biológica de um composto químico fora de um conjunto de treino, depende
de muitos fatores como a qualidade dos dados e as escolhas dos descritores mais significantes. Em modelos \ac{QSAR}, utilizam-se geralmente duas 
categorias de desctirores moleculares: descritores químicos bidimensionais, baseados na representação bidimensional dos compostos, e descritores 
químicos tridimensionais \citep{todeschini2000handbook}.

As técnicas de \ac{ML} têm sido muito aplicadas ao desenvolvimento de modelos \ac{QSAR}. 
Por exemplo, o método dos mínimos quadrados adaptativos difusos (\emph{fuzzy adaptive least squares}) é utilizado em \citep{Yoshida2000}
para treinar um modelo \ac{QSAR} para classificar os fármacos em uma das $4$ classes predifinidas de biodisponibilidade de acordo com a presença ou
ausência de grupos funcionais típicos mais suscetíveis de estarem envolvidos em reações metabólicas. Em \citep{pintore2003prediction},
\acp{AG} foram utilizados para selecionar os melhores descritores moleculares e mapas auto-organizáveis (\ac{SOM}) 
foram utilizados para atribuir uma classe de biodisponibilidade à cada um. Em
\citep{dureja2008topological} foram utilizadas
árvores aleatórias (\ac{RF}), árvores de decisão e o método de médias móveis para a previsão dos parâmetros farmacocinéticos
da cefalosporina\footnote{A cefalosporina é um antibiótico, semelhante a penincilina, utilizado no tratamento de infeções bacterianas.}.

Máquinas de vetores de suporte (\ac{SVM}) são utilizadas em \citep{frohlich2006kernel} para a previsão de 
biodisponibilidade estimando a similaridade entre moléculas diferentes com comportamentos biológicos semelhantes. 

As redes neuronais artificiais (\ac{ANN}) são geralmente utilizadas para a construção de modelos \ac{QSAR} \citep{zupan1999neural} e são frequentemente
integradas em pacotes de \emph{software} comerciais de empresas envolvidas no ramo de modelação molecular. Entre os líderes neste ramo,
as empresas Accelrys Inc. \citep{accelrys2004modeling} e PharmaAlgorithms Inc. \citep{pharma2006adme} provêm modelos matemáticos de 
caixa-preta\footnote{Sistemas fechados de complexidade muito alta em que a sua estrutura interna (neste caso fórmula matemática) é desconhecida.}
e ferramentas de \emph{data mining} para a construção de novos indicadores.


Os \ac{AE} têm sido utilizados com enorme sucesso na elaboração de modelos moleculares 
\citep{clark1996evolutionary} e em outras tarefas no processo de desenvolvimento de fármacos \citep{lameijer2005evolutionary}. 
Por exemplo em \citep{ordog2008evaluating}, \acp{AG} foram utilizados para avaliar a 
velocidade de convergência e as propriedades de desvio na otimização da energia de acoplamento molecular produzidas pelo 
\emph{software} de código-livre Autodock 3.05 do \emph{Scripps Research Institute} \citep{morris1998automated}. Em 
\citep{wegner2003prediction}, os \acp{AG} foram utilizados para a redução/seleção de características de um modelo \ac{QSAR} 
para a previsão da soludibilidade de um fármaco na água.

Nos últimos anos, a \ac{PG} tem-se tornado uma escolha popular em modelação \ac{QSAR} e em aplicações biomédicas relacionadas. Por exemplo,
em \citep{langdon2005genetic}, a \ac{PG} foi utilizada para classificar moléculas em termos da sua biodisponibilidade; 
em \citep{Archetti2007,vanneschi2009using,vanneschi2010measuring} e \citep{castelli2011quantitative} foi utilizada para 
a previsão da biodisponibilidade oral, 
do grau de ligação as proteínas do plasma e da toxicidade induzida em fármacos. Em \citep{archetti2010genetic}, a \ac{PG} foi utilizada para gerar uma relação funcional
entre um conjunto de descritores moleculares e a sua energia de acoplamento molecular.

Em \citep{yu2007feature}, a \ac{PG} é aplicada a dados de perfis de expressões de câncer para selecionar os atributos (\emph{features}) dos genes
e construir classificadores moleculares, através da integração matemática dos mesmos genes. A \ac{PG} foi também utilizada em conjunto com
a base de dados NCI-60 para procurar uma relação entre as expressões génicas e a sua resposta aos medicamentos oncológicos (fluoruracila, 
fludarabina, floxuridina e citarabina) com o objetivo de determinar a probalidade de resistência aos mesmos.


A previsão eficaz dos parâmetros farmacocinéticos e outros componentes do processo de desenvolvimento de medicamentos foi também tratada
no presente trabalho, estabelecendo uma comparação com os resultados apresentados na bibliografia, com realce aos descritos em 
\citep{Archetti2007,vanneschi2009using,vanneschi2010measuring,castelli2011quantitative,archetti2010genetic}
e \citep{archetti2010genetic}.
Para tal, foram aplicadas abordagens diferentes, descritas no capítulo \ref{cap:5}, com a intenção de avaliar a performance da \ac{PG}
na resolução do problema em questão e descobrir o quanto os resultados podem ser melhorados.