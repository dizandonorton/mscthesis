\section{INTRODUÇÃO}
\label{sec:3introducao}

Os medicamentos\footnote{No presente trabalho, os termos \emph{medicamento}, \emph{fármaco} e \emph{droga} serão utilizados alternadamente.} 
receitados para lidar com certas doenças podem por vezes produzir efeitos colaterais sobre o corpo humano\footnote{Por questões 
de simplicidade aqui nos referimos ao corpo humano. No entanto esses conceitos podem ser generalizados à outros animais.}. 
Esses efeitos ou reações adversas podem acontecer nalgum ponto crítico do ciclo de vida do medicamento. Sendo assim, é importante
perceber quais propriedades químicas dos medicamentos produzem o efeito desejado. Este objetivo é alcançado pelo processo de triagem de 
alta produtividade (\ac{HTS}), que é um método computacional para a busca de moléculas relevantes
em enormes quantidades de substâncias que compôem os medicamentos \citep{TDI2013}.

Os resultados obtidos pela triagem de alta produtividade são posteriormente utilizados para otimizar as propriedades de certas moléculas
no processo de fabricação de medicamentos. Além disso, é necessário garantir que os fármacos percorram o caminho apropriado no corpo humano
sem alterar a saúde do paciente. Por esta razão, o comportamento das moléculas é avaliado durante os processos de absorção, distribuição, 
metabolismo e excreção do medicamento no organismo. O estudo deste processo recebe o nome de farmacocinética \citep{DiPiro2010}. 

Em suma, a farmacocinética dá uma resposta a pergunta sobre como o corpo lida com o fármaco e é composta pelas seguintes fases:
\begin{itemize}
  \item {\textbf{Absorção} - é o processso de entrada das substâncias na circulação sanguínea}
  \item {\textbf{Distribuição} - é a dispersão ou disseminaçao das substâncias através dos fluídos e tecidos do corpo}
  \item {\textbf{Metabolismo} - é o reconhecimento, pelo organismo, de que uma substância está presente e a transformação irreversível dos compontes desta
  substância em metabolitos}
  \item {\textbf{Excreção} - é a remoção das substãncias do corpo. Raramente alguns medicamentos acumulam-se nos tecidos do corpo}
  \item {\textbf{Toxicidade} - representa o quanto uma substância pode prejudicar o organismo}
\end{itemize}

Quase metade das falhas no desenvolvimento de compostos fármacológicos são relacionadas com a farmacocinética e a toxicidade dos fármacos,
segundo estudos publicados nos anos de 1991 \citep{Kennedy1997} e 2000 \citep{Kola2004}, tal como ilustrado na \figref{Figura311}.

\begin{figure}[H]
	\centering
	\begin{tikzpicture}[scale = 0.75]
		%\pie[color={blue,cyan,}]{10/, 30/, 0/, 39/, 5/, 11/, 0/, 5/}
		\pie[color={violet!70, cyan, orange, red!70, blue!70, brown!40}]
		{10/, 30/, 39/, 5/, 11/, 5/}
		\pie[pos={7,0}, text=legend, color={violet!70, cyan, yellow, orange, red!70, blue!70, gray, brown!40}]
		{12/Clínicas, 26/Eficâcia, 4/Elaboração, 7/Farmacocinética, 19/Comerciais, 19/Toxicológicos, 7/Custo dos produtos, 6/Outros}
	\end{tikzpicture}
	\caption{Razões para falhas no desenvolvimento de fármacos (valores aproximados). O gráfico da direita ilustra os resultados de 2009
	e o da esquerda os resultados de 2000}
	\label{Figura311}
\end{figure}

Alguns parâmetros farmacocinéticos diretamente correlacionados com o processo \ac{ADMET} serão utilizados neste trabalho, nomeadamente:
a Biodisponiblidade Oral \ac{F}, o nível de Ligação às Proteínas do Plasma (\ac{PPB}) e a Dose Letal Mediana \ac{LD50}\footnote{Para uma lista mais 
extensa dos parâmetros farmacocinéticos, consulte \citep{blode2004collection}.}. Outro parâmetro diretamente ligado ao comportamento dos fármacos no 
organismo, e que será utilizado no presente trabalho, é a energia de acoplamento molecular. Finalmente, também será utilizado o fármaco Fludarabina. As
próximas secções apresentam uma breve descrição destes itens e como serão utilizados no presente trabalho.
