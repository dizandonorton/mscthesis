\section{PARÂMETROS FARMACOCINÉTICOS UTILIZADOS}
\label{sec:3parametros}

\subsection{\ac{F}}

A \ac{F} indica a percentagem do medicamento administrado que chega ao sistema circulatório comparada com o método de 
administração intravenoso\footnote{A administração intravenosa consiste na injeção de medicamentos por meio de 
agulhas nas veias periféricas dos membros superiores.} após a passagem pelo fígado. A biodisponilidade oral é determinada pelos de 
processos farmacocinéticos de \emph{absorção} e \emph{metabolismo}. Após uma administração intravenosa, a biodisponibilidade é $100\%$ enquanto que numa 
administração oral a biodisponibilidade é geralmente inferior. Tipicamente, isto acontece devido a muitos fatores, sendo um deles a incompleta absorção 
intestinal \citep{Urso2002}.

\subsection{\ac{PPB}}

A distribuição de um medicamento do plasma para os tecidos de destino no corpo humano pode ser afetada por vários fatores e um dos mais 
importantes é a percentagem de ligação às proteínas do plasma (\ac{PPB}). Quanto menor for essa percentagem, mais eficientemente 
o fármaco atravessa as membranas celulares e se difunde. No sangue, uma proporção de um fármaco pode
estar ligada ou não ligada, de acordo a sua afinidade às proteínas do plasma \citep{shargel2005}. 
A proporção não-ligada é a que possui efeitos farmacológicos e é também esta porção que será metabolizada e/ou excretada. Por exemplo, a proporção ligada do 
anticoagulante varfarina\footnote{A varfarina é um anticoagulante utilizado na prevenção de tromboses.} é de $97\%$. Isto significa que 
a quantidade de varfarina no sangue, $97\%$, está ligada à proteínas do plasma. O restante $3\%$ (fracção não-ligada) é a fracção não-ativa e poderá ser excretada
\citep{shargel2005}. Substâncias que se ligam fortemente às proteínas do plasma têm grande impacto na eficâcia do fármaco 
uma vez que são as responsáveis pela acção do mesmo. 

\subsection{\ac{LD50}}

A \ac{LD50} é um teste feito para determinar o risco ou potencial de toxicidade de substâncias existentes ou novas. 
Os testes de \ac{LD50}, geralmente realizados em ratos, são caros, morosos e ativamente combatidos por ativistas.
Em particular, a informação sobre a toxicidade aguda de substâncias químicas é necessária como um dos critérios essenciais para 
avaliar sua segurança \citep{Devillers2009}.

\subsection{Energia de acoplamento molecular}

O objetivo do acoplamento molecular no desenvolvimento de fármacos é de identificar drogas candidatas direcionadas às proteínas receptoras
no organismo. Estas drogas candidatas podem ser encontradas utilizando um algoritmo de acoplamento que tenta identificar 
uma ligação otimizada de uma pequena molécula (chamada de \emph{ligante}) às proteínas receptoras no seu local ativo de formas 
que a energia livre de todo o sistema seja minimizada. Esta energia é chamada de: energia de acoplamento molecular \citep{Thomsen2007}.

\subsection{Fludarabina}

A fludarabina não é um parâmetro farmacocinético mas sim um fármaco utilizado no tratamento de leucemia linfocítica\footnote{A 
leucemia é uma neoplasia que afecta os glóbulos brancos (células imaturas da medula óssea).}.  
É um dos $118$ fármacos que fazem parte da base de dados NCI-60 do \ac{NCI}\footnote{Instituto Nacional do Câncer dos Estados Unidos da América: 
\url{www.cancer.gov}.} \citep{nci2008} que é composta por $60$ linhagens de células tumorais derivadas 
de pacientes com os seguintes $9$ tipos de câncer: colo-retal, renal, do ovário, da mama, da próstata, do sistema nervoso 
central, leucemias e melanomas \citep{del2007gene}.

A acção da fludarabina (e de outros fármacos) foi estimada através da medição da inibição do crescimento das linhagens de células tumorais
$48$ horas após tratamento e é definida como sendo a concentração logarítmica necessária para reduzir 
a taxa de crescimento para $50\%$. O nosso objetivo é encontrar uma relação matemática entre o perfil das expressões génicas e o padrão 
de atividade da fludarabina.